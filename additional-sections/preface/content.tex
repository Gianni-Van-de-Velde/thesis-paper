
\titleformat{\chapter}{}{}{0em}{\bf\Huge}
\chapter*{Preface}
\addcontentsline{toc}{chapter}{Preface}

For this thesis, I had the opportunity to write my own proposal. I wanted to work on the energy efficiency of LLMs and more specifically in RAG scenarios. Needless to say, I am very excited about the topic. There are three passions of mine, that I could combine in this thesis: a passion for AI, efficiency and environmental friendliness. 

For the AI part, I got to work with LLMs, to find out how they work internally and to improve their way of working. This further sparked my interest in AI and made clear to me that I really want to work in that domain. Next, the collaboration with the UZGent, allowed me to work with real world impact. Next to the theoretical improvements, I also aspired to work on a thesis that will measurably improve the lifes of people. To do so, I tried to improve the efficiency of searching information in a database at the UZGent. Each minute that my work can save a healthcare professional is a minute that they can help people. Lastly, I try to be mindful about my environmental impact in all aspects of my life. So combining the positive impact for the UZGent with reducing the energy required to do so was a perfect combination for me. I explored methods that allow reducing the energy consumption of LLMs, reducing the energy necessary with about a factor TODO. This maked the implementation more environmental friendly in one case, but the same principle could be used by any other LLM host, increasing the potential impact massively.
