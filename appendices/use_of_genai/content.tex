\chapter{Use of Generative AI}

The responsible use of AI was allowed for this thesis. Responsible use is defined as being aware of the limitations, risks, pitfalls of using generative AI, such as (un)reliability, bias, copyright, data security, privacy, social inequality, ethical and sustainable impact, etc. \cite{ugent2025verantwoord}.

For this thesis, only boilerplate code was generated with generative AI. This improved coding efficiency, while being low risk. The boilerplate code is quite predictable and easily checked for errors (which I always did). For data security, no sensitive data was given to the generative AI tools. The sustainable impact was limited as much as possible, by keeping the generated answers short, which limits the energy consumption \cite{poddar2025towards}. We did this with prompt additions such as ``Answer shortly.'', ``Give code only.'' and ``Only reply with the relevant part of the code.''
