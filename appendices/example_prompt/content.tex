\chapter{Example prompt of UZGentRAG}
\label{sec:example_prompt}

We introduced a new private benchmark, UZGentRAG. In order to give some idea what a typical UZGentRAG prompt might look like, one example is given below. It always starts with a role description. Then the documents are given, split by ``---''. Finally, the question of the user is added.

\begin{wrapverbatim}
[begin context]
Jij bent een dokter. Jij kent alle medische procedures van het UZGent.
[einde context]
[begin documenten]
obij verhoogde nuchtere of at random glycemie: bijkomend bepalen HbA1c
 
4.5 Diagnostiek cont'd 
Beeldvorming 
-Hypofyse:
oI.g.v. anders onverklaarde hoofdpijn, gezichtsvelduitval (met name uitval lateraal), 
hormonaal bilan suggestief voor hypofysitis / hypopituïtarisme : dringende MRI hypofyse 
(hypofysitis? RIP?), in overleg met MRI radioloog van wacht.
oZo normale MRI en sterk suggestief klinisch beeld, nieuwe MRI na 1 maand.
 
-Schildklier:
oBij thyreotoxicosis, pijnlijke schildklieropzetting - te plannen via polikliniek Endocrinologie 
(urgente poli-afspraak via D22137, na overleg met endocrinoloog van wacht).
5. Beleid en verloop
5.1 Behandeling hypothyreose
Graad
 
Graad 1
(TSH < 10 mU/L): 
 
Graad 2-3
(TSH > 10 mU/L ): Graad 4
 
 
Behandeling
 
-Controle na 6 weken;

---

-Labo:
oHbA1c
oBloedgroep en Indirecte Coombs
oEPD >  Plaatsen van Orders > Subset Klinische biologie Gyn/Materniteit > Prenataal 1ste 
consult
o(NB: enkel bepalen wat nog niet bekend is).
oZo combinatietest gewenst: EPD >  Plaatsen van Orders > Subset Klinische biologie 
Gyn/Materniteit > Combinatietest 1ste trimester
oZo bekende schildklierpathologie: (zie richtlijn "Schildklierlijden en zwangerschap")
-Bepaal ook FT4, FT3.
-Bij gekende Morbus Graves: bepaal TSI-antilichamen (belang voor foetale en neonatale 
opvolging).
-Bij hypothyreoïdie en bekende diagnose Morbus Graves: geen verdere bepalingen van 
antistoffen nodig.
-Bij onduidelijke diagnose: bepaal TPO- en TSI-antilichamen.
-Bij gekende hypothyreoïdie: verhoog dosis levothyroxine met 25 tot 50\%

---

- Inname levothyroxine en zwangerschapsvitamines/ijzersuppletie met minstens 2 uur 
tussen (ivm verminderde absorptie bij gebruik met ijzerbevattende supplementen)
Nieuwe diagnose hypothyreoïdie
- Zie flowchart nieuwe diagnose hypothyreoïdie
- Indien substitutie aangewezen is, start een lage dosis levothyroxine bvb 50mcg per 
dag.
Gekende hypothyreoïdie
- Anamnese bij 1ste consultatie: Hoe lang bestaand? Ontstaan? Oorzaak? Uitsluiten 
behandelde Morbus Graves. Behandeling?
- Bij een positieve zwangerschapstest dosis met 25\% verhogen (dit kan vb. door 2 van 
de 7 dagen van de week een dubbele dosis in te nemen).
- Controle bij 1e bloedafname in de zwangerschap:
- Bepaling van TSH, vrij-T4 en vrij-T3
- Bepaling van TSI i.g.v. iatrogene hypothyreoïdie bij ziekte van Graves of bij
[einde documenten]
[begin vraag]
Wat moet ik doen bij een graad 1 immuun-gerelateerde hypothyroïdie?
[einde vraag]
[begin antwoord]

\end{wrapverbatim}
