\chapter{Deliverables to UZGent}

In this thesis we worked together with UZGent and we made sure both parties benefited from the collaboration. In this section, we list the documentation, code and data that we deliver to UZGent.

\section{Documentation}
The RAG application ran on the servers of UZGent. However, after the project the server was reclaimed for other purposes. This means the setup needs to happen again if UZGent decides to go further with the application and to help with that, we logged all commands that are necessary to set up the server again. After following the instructions, the requirements are installed, https is set up, the RAG API is live and the website is hosted on that server. This all works with the building blocks explained in Figure \ref{fig:architecture_docker}. The only thing that still needs to happen is to contact the networking person to open up the right ports of that server to the internal network of UZGent.

\section{Code}
The setup above requires access to the code on two GitHub repositories. One repository is for the web application. The other repository contains the code to run the RAG API, which also comes with a short demo notebook that shows how the different components work. We grant access to the ICT department of UZGent on demand. 

\section{Data}
We held a test session to gather typical questions from healthcare professionals and the corresponding document in Zenya. This data shows how the system is currently used and allows for further ``in vitro'' experiments (as the experts of UZGent like to call it). By that, we mean that the usage of Zenya is not too dynamic and thus the test set can be used for experiments in the coming year(s), without further input required.

The most straightforward use of this data is to evaluate retrievers, in order to find the optimal way of presenting the Zenya data to healthcare professionals. To do this properly, we split the data in a validate set and a test set. The provider of the retriever can use the validate set to tune its retriever specifically for the use case. Then, the UZGent can test the retriever with their withheld test set. Both the validate and test set were delivered in the form of CSVs, as this was most practical for UZGent.

