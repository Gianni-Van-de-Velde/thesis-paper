
\chapter{Societal impact}

In this thesis, many societal aspects are touched. Firstly, the techniques used seek to reduce the energy usage of LLMs in specific use cases. Secondly, the same methods allow to reduce latency of LLM generation, the impact of which can range very wide. Finally, a novel benchmark can boost the research in the field (TODO does it?).

\section{Reducing energy consumption of LLMs}
% https://knowsdgs.jrc.ec.europa.eu/themes/sdgs/assets/img/sdg13.png
% https://encrypted-tbn0.gstatic.com/images?q=tbn:ANd9GcQk1PBOYL87cU1pc5NFBd3KD9oby7ecHBF_jw&s

In recent years, the size of language models has increased exponentially. This raised already existing concerns about the energy usage of those language models (by now LLMs). Major players in the market have start building nuclear power plants, with the sole purpose to satisfy the energy demand increase by larger models. Needless to say, every percentage of the total energy cost that could be decreased will have immense impact. This is what is made possible with speculative decoding. Though it addresses a niche of the wide range of LLM applications, it can make significant improvements for such use cases. 

So what is the societal impact? It is no surprise that the greenhouse gas emissions need to be reduced globally. Part of the emissions come from energy production and thus reducing energy usage can help society achieving their green ambitions. However, a common pitfall is to ignore the rebound effect when improving energy efficiency. Applying it to this thesis: with the reduced energy consumption, the host of the LLM might consider using a larger model, which would nullify the energy reduction. 

\section{Reducing LLM latency}
% https://upload.wikimedia.org/wikipedia/commons/thumb/b/bc/Sustainable_Development_Goal_03GoodHealth.svg/800px-Sustainable_Development_Goal_03GoodHealth.svg.png

% https://knowsdgs.jrc.ec.europa.eu/themes/sdgs/assets/img/sdg4.png

With the wide range of applications for LLMs, reducing latency can have a whole array of impacts. For example many students augment their learning experience, by asking LLMs to explain course content. However,this thesis focusses on a use case for UZGent. The goal is to make a large set of documents more approachable with a chatbot that answers questions based on the documents. This reduces the time healthcare professionals spend searching in a database and thus increases the time they can spend on helping people. So in a sense, this thesis contributes to SDG3: good health and well-being.

\section{A new benchmark for TODO}
Benchmarks are the cornerstone of research in the domain of LLMs. Each benchmark has its own strengths and weaknesses. The strength of the new benchmark is that it combines three aspects. First of all, it is based on a real world use case. Often times, benchmarks are quite artificial, growing a gap between real world performance and benchmark performance. This effect should be absent when the benchmark is made using a real world use case. Secondly, the benchmark is in a specific domain: the domain of healthcare. This brings new challenges, often forgotten with general benchmarks. Lastly, a niche language was used: Dutch. In conclusion, this is a niche benchmark. Niche benchmarks allow fair comparison between different methods in specific scenarios, rather than the most general scenarios.

